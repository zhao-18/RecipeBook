%% Add after the wrapping text
\def\wrapfill{
    \par
  \ifx\parshape\WF@fudgeparshape
    \nobreak
    \ifnum\c@WF@wrappedlines>\@ne
      \advance\c@WF@wrappedlines\m@ne
      \vskip\c@WF@wrappedlines\baselineskip
      \global\c@WF@wrappedlines\z@
    \fi
    \allowbreak
    \WF@finale
  \fi
}


%% Used for the headnote and in \showit
%% If the text is small it is placed on one line;
%% otherwise it is put into a raggedright paragraph.
\long\def\testoneline#1{%
  \sbox\@tempboxa{#1}%
  \ifdim \wd\@tempboxa <0.75\linewidth
        \begingroup
            \itshape
            #1\par
        \endgroup
  \else
    \parbox{0.75\linewidth}{\raggedright\itshape#1}%
    \par
  \fi
}

%% #1 Title of recipe; #2 [Initial instructions]
\NewDocumentCommand{\recipe}{m o}{%
    \setcounter{stepnum}{0}%
    \newpage
    \lhead{}%
    \chead{}%
    \rhead{}%
    \lfoot{}%
    \rfoot{}%
    \phantomsection
    \addcontentsline{toc}{subsection}{#1}%
    \subsection*{#1}%
    \IfNoValueF{#2}{\noindent\emph{#2}\par\medskip}
}

%% Temperature
\newcommand{\temp}[1]{#1°F}

%% Control highlighting ingredients
%% Ingredient first, then measure; empty measure and/or unit = " . "
\def\ucit#1{\uppercase{#1}}
\begingroup
    \lccode`~=`\^^M
    \lowercase{%
\endgroup
    %% *=column break; amount<space>ingredient
    \NewDocumentCommand{\ing}{u{ } u{ } u{~}}{% %% basically the same as: \def\ing#1 #2~{% requires xparse
        \noindent
        \if#1#2% Is a heading, a non-ingredient, in the ingredients block
            \emph{#3}~ % A heading
        \else % Amounts containing spaces <1 teaspoon> have to use '~' <1~teaspoon>
            \textbf{\ucit#3, }#1\if.#2\else\ #2\fi~ %
        \fi
    }%
}%

\NewDocumentEnvironment{step}{}{%
    \parindent0pt
    \leftskip0pt
    \begin{minipage}{\textwidth}
        \begin{wrapfigure}{r}{0pt}
            \kern-0.5em
            \vrule width 1pt\enskip
            \begin{minipage}{0.5\textwidth}
                \leftskip=1.5em
                \parindent=-1.5em
                \parskip=0.25em
                \obeylines
                    \everypar={\ing}
}{%
        \wrapfill
    \end{minipage}
    \medskip
}

%Recipe steps on left side
\NewDocumentCommand{\method}{}{%
            \end{minipage}
        \end{wrapfigure}
        \rightskip0pt plus 2em
        \parskip1.0em
        \everypar={\llap{\stepcounter{stepnum}\hbox to 1.5em{\thestepnum.\hfill}}}
}

\titleformat{\section}{\normalfont\Large\bfseries}{\thesection.}{0.5em}{\centering}
